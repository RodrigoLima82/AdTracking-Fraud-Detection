\documentclass[]{article}
\usepackage{lmodern}
\usepackage{amssymb,amsmath}
\usepackage{ifxetex,ifluatex}
\usepackage{fixltx2e} % provides \textsubscript
\ifnum 0\ifxetex 1\fi\ifluatex 1\fi=0 % if pdftex
  \usepackage[T1]{fontenc}
  \usepackage[utf8]{inputenc}
\else % if luatex or xelatex
  \ifxetex
    \usepackage{mathspec}
  \else
    \usepackage{fontspec}
  \fi
  \defaultfontfeatures{Ligatures=TeX,Scale=MatchLowercase}
\fi
% use upquote if available, for straight quotes in verbatim environments
\IfFileExists{upquote.sty}{\usepackage{upquote}}{}
% use microtype if available
\IfFileExists{microtype.sty}{%
\usepackage{microtype}
\UseMicrotypeSet[protrusion]{basicmath} % disable protrusion for tt fonts
}{}
\usepackage[margin=1in]{geometry}
\usepackage{hyperref}
\hypersetup{unicode=true,
            pdftitle={Projeto01.R},
            pdfauthor={rodrigolima82},
            pdfborder={0 0 0},
            breaklinks=true}
\urlstyle{same}  % don't use monospace font for urls
\usepackage{color}
\usepackage{fancyvrb}
\newcommand{\VerbBar}{|}
\newcommand{\VERB}{\Verb[commandchars=\\\{\}]}
\DefineVerbatimEnvironment{Highlighting}{Verbatim}{commandchars=\\\{\}}
% Add ',fontsize=\small' for more characters per line
\usepackage{framed}
\definecolor{shadecolor}{RGB}{248,248,248}
\newenvironment{Shaded}{\begin{snugshade}}{\end{snugshade}}
\newcommand{\AlertTok}[1]{\textcolor[rgb]{0.94,0.16,0.16}{#1}}
\newcommand{\AnnotationTok}[1]{\textcolor[rgb]{0.56,0.35,0.01}{\textbf{\textit{#1}}}}
\newcommand{\AttributeTok}[1]{\textcolor[rgb]{0.77,0.63,0.00}{#1}}
\newcommand{\BaseNTok}[1]{\textcolor[rgb]{0.00,0.00,0.81}{#1}}
\newcommand{\BuiltInTok}[1]{#1}
\newcommand{\CharTok}[1]{\textcolor[rgb]{0.31,0.60,0.02}{#1}}
\newcommand{\CommentTok}[1]{\textcolor[rgb]{0.56,0.35,0.01}{\textit{#1}}}
\newcommand{\CommentVarTok}[1]{\textcolor[rgb]{0.56,0.35,0.01}{\textbf{\textit{#1}}}}
\newcommand{\ConstantTok}[1]{\textcolor[rgb]{0.00,0.00,0.00}{#1}}
\newcommand{\ControlFlowTok}[1]{\textcolor[rgb]{0.13,0.29,0.53}{\textbf{#1}}}
\newcommand{\DataTypeTok}[1]{\textcolor[rgb]{0.13,0.29,0.53}{#1}}
\newcommand{\DecValTok}[1]{\textcolor[rgb]{0.00,0.00,0.81}{#1}}
\newcommand{\DocumentationTok}[1]{\textcolor[rgb]{0.56,0.35,0.01}{\textbf{\textit{#1}}}}
\newcommand{\ErrorTok}[1]{\textcolor[rgb]{0.64,0.00,0.00}{\textbf{#1}}}
\newcommand{\ExtensionTok}[1]{#1}
\newcommand{\FloatTok}[1]{\textcolor[rgb]{0.00,0.00,0.81}{#1}}
\newcommand{\FunctionTok}[1]{\textcolor[rgb]{0.00,0.00,0.00}{#1}}
\newcommand{\ImportTok}[1]{#1}
\newcommand{\InformationTok}[1]{\textcolor[rgb]{0.56,0.35,0.01}{\textbf{\textit{#1}}}}
\newcommand{\KeywordTok}[1]{\textcolor[rgb]{0.13,0.29,0.53}{\textbf{#1}}}
\newcommand{\NormalTok}[1]{#1}
\newcommand{\OperatorTok}[1]{\textcolor[rgb]{0.81,0.36,0.00}{\textbf{#1}}}
\newcommand{\OtherTok}[1]{\textcolor[rgb]{0.56,0.35,0.01}{#1}}
\newcommand{\PreprocessorTok}[1]{\textcolor[rgb]{0.56,0.35,0.01}{\textit{#1}}}
\newcommand{\RegionMarkerTok}[1]{#1}
\newcommand{\SpecialCharTok}[1]{\textcolor[rgb]{0.00,0.00,0.00}{#1}}
\newcommand{\SpecialStringTok}[1]{\textcolor[rgb]{0.31,0.60,0.02}{#1}}
\newcommand{\StringTok}[1]{\textcolor[rgb]{0.31,0.60,0.02}{#1}}
\newcommand{\VariableTok}[1]{\textcolor[rgb]{0.00,0.00,0.00}{#1}}
\newcommand{\VerbatimStringTok}[1]{\textcolor[rgb]{0.31,0.60,0.02}{#1}}
\newcommand{\WarningTok}[1]{\textcolor[rgb]{0.56,0.35,0.01}{\textbf{\textit{#1}}}}
\usepackage{graphicx,grffile}
\makeatletter
\def\maxwidth{\ifdim\Gin@nat@width>\linewidth\linewidth\else\Gin@nat@width\fi}
\def\maxheight{\ifdim\Gin@nat@height>\textheight\textheight\else\Gin@nat@height\fi}
\makeatother
% Scale images if necessary, so that they will not overflow the page
% margins by default, and it is still possible to overwrite the defaults
% using explicit options in \includegraphics[width, height, ...]{}
\setkeys{Gin}{width=\maxwidth,height=\maxheight,keepaspectratio}
\IfFileExists{parskip.sty}{%
\usepackage{parskip}
}{% else
\setlength{\parindent}{0pt}
\setlength{\parskip}{6pt plus 2pt minus 1pt}
}
\setlength{\emergencystretch}{3em}  % prevent overfull lines
\providecommand{\tightlist}{%
  \setlength{\itemsep}{0pt}\setlength{\parskip}{0pt}}
\setcounter{secnumdepth}{0}
% Redefines (sub)paragraphs to behave more like sections
\ifx\paragraph\undefined\else
\let\oldparagraph\paragraph
\renewcommand{\paragraph}[1]{\oldparagraph{#1}\mbox{}}
\fi
\ifx\subparagraph\undefined\else
\let\oldsubparagraph\subparagraph
\renewcommand{\subparagraph}[1]{\oldsubparagraph{#1}\mbox{}}
\fi

%%% Use protect on footnotes to avoid problems with footnotes in titles
\let\rmarkdownfootnote\footnote%
\def\footnote{\protect\rmarkdownfootnote}

%%% Change title format to be more compact
\usepackage{titling}

% Create subtitle command for use in maketitle
\providecommand{\subtitle}[1]{
  \posttitle{
    \begin{center}\large#1\end{center}
    }
}

\setlength{\droptitle}{-2em}

  \title{Projeto01.R}
    \pretitle{\vspace{\droptitle}\centering\huge}
  \posttitle{\par}
    \author{rodrigolima82}
    \preauthor{\centering\large\emph}
  \postauthor{\par}
      \predate{\centering\large\emph}
  \postdate{\par}
    \date{2019-05-19}


\begin{document}
\maketitle

\begin{Shaded}
\begin{Highlighting}[]
\CommentTok{# Script para checar as colunas do dataset}

\CommentTok{# Carregando os Pacotes}
\KeywordTok{library}\NormalTok{(data.table)}

\CommentTok{# Carregando o Dataset # }
\CommentTok{# Os dados brutos contém 100.000 linhas e 8 colunas (atributos). }
\CommentTok{# A coluna "is_attributed" é o alvo.}
\NormalTok{dt <-}\StringTok{ }\KeywordTok{fread}\NormalTok{(}\StringTok{"dados/train_sample.csv"}\NormalTok{)}
\NormalTok{df <-}\StringTok{ }\KeywordTok{as.data.frame}\NormalTok{(dt)}

\CommentTok{# Remove dt}
\KeywordTok{rm}\NormalTok{(}\StringTok{'dt'}\NormalTok{)}

\CommentTok{# Visualizando dados do dataframe}
\CommentTok{# View(df)}
\KeywordTok{str}\NormalTok{(df)}
\end{Highlighting}
\end{Shaded}

\begin{verbatim}
## 'data.frame':    100000 obs. of  8 variables:
##  $ ip             : int  87540 105560 101424 94584 68413 93663 17059 121505 192967 143636 ...
##  $ app            : int  12 25 12 13 12 3 1 9 2 3 ...
##  $ device         : int  1 1 1 1 1 1 1 1 2 1 ...
##  $ os             : int  13 17 19 13 1 17 17 25 22 19 ...
##  $ channel        : int  497 259 212 477 178 115 135 442 364 135 ...
##  $ click_time     : chr  "2017-11-07 09:30:38" "2017-11-07 13:40:27" "2017-11-07 18:05:24" "2017-11-07 04:58:08" ...
##  $ attributed_time: chr  "" "" "" "" ...
##  $ is_attributed  : int  0 0 0 0 0 0 0 0 0 0 ...
\end{verbatim}

\begin{Shaded}
\begin{Highlighting}[]
\CommentTok{# Nome das variáveis}
\CommentTok{# ip, app, device, os, channel, click_time, attributed_time, is_attributed}

\CommentTok{# Aplicando Engenharia de Atributos em Variaveis Numericas}

\CommentTok{# Transformar o objeto de data}
\NormalTok{df}\OperatorTok{$}\NormalTok{click_time <-}\StringTok{ }\KeywordTok{as.POSIXct}\NormalTok{(df}\OperatorTok{$}\NormalTok{click_time)}

\CommentTok{# Extraindo dia e hora}
\NormalTok{df}\OperatorTok{$}\NormalTok{click_day <-}\StringTok{ }\KeywordTok{as.integer}\NormalTok{(}\KeywordTok{format}\NormalTok{(df}\OperatorTok{$}\NormalTok{click_time, }\StringTok{"%d"}\NormalTok{))}
\NormalTok{df}\OperatorTok{$}\NormalTok{click_hour <-}\StringTok{ }\KeywordTok{as.integer}\NormalTok{(}\KeywordTok{format}\NormalTok{(df}\OperatorTok{$}\NormalTok{click_time, }\StringTok{"%H"}\NormalTok{))}

\CommentTok{# Transformando variáveis numéricas em variáveis categóricas}
\NormalTok{df}\OperatorTok{$}\NormalTok{is_attributed <-}\StringTok{ }\KeywordTok{as.factor}\NormalTok{(df}\OperatorTok{$}\NormalTok{is_attributed)}


\CommentTok{# Remover colunas nao utilizadas}
\NormalTok{df}\OperatorTok{$}\NormalTok{click_time <-}\StringTok{ }\OtherTok{NULL}
\NormalTok{df}\OperatorTok{$}\NormalTok{attributed_time <-}\StringTok{ }\OtherTok{NULL}

\KeywordTok{str}\NormalTok{(df)}
\end{Highlighting}
\end{Shaded}

\begin{verbatim}
## 'data.frame':    100000 obs. of  8 variables:
##  $ ip           : int  87540 105560 101424 94584 68413 93663 17059 121505 192967 143636 ...
##  $ app          : int  12 25 12 13 12 3 1 9 2 3 ...
##  $ device       : int  1 1 1 1 1 1 1 1 2 1 ...
##  $ os           : int  13 17 19 13 1 17 17 25 22 19 ...
##  $ channel      : int  497 259 212 477 178 115 135 442 364 135 ...
##  $ is_attributed: Factor w/ 2 levels "0","1": 1 1 1 1 1 1 1 1 1 1 ...
##  $ click_day    : int  7 7 7 7 9 9 9 7 8 8 ...
##  $ click_hour   : int  9 13 18 4 9 1 1 10 9 12 ...
\end{verbatim}

\begin{Shaded}
\begin{Highlighting}[]
\CommentTok{# Analise Exploratoria de Dados}

\CommentTok{# Carregando os Pacotes}
\KeywordTok{library}\NormalTok{(dplyr)}
\end{Highlighting}
\end{Shaded}

\begin{verbatim}
## 
## Attaching package: 'dplyr'
\end{verbatim}

\begin{verbatim}
## The following objects are masked from 'package:data.table':
## 
##     between, first, last
\end{verbatim}

\begin{verbatim}
## The following objects are masked from 'package:stats':
## 
##     filter, lag
\end{verbatim}

\begin{verbatim}
## The following objects are masked from 'package:base':
## 
##     intersect, setdiff, setequal, union
\end{verbatim}

\begin{Shaded}
\begin{Highlighting}[]
\CommentTok{# Verificar se existem valores ausentes (missing) em cada coluna}
\CommentTok{# Nenhum valor encontrado}
\KeywordTok{any}\NormalTok{(}\KeywordTok{is.na}\NormalTok{(df))}
\end{Highlighting}
\end{Shaded}

\begin{verbatim}
## [1] FALSE
\end{verbatim}

\begin{Shaded}
\begin{Highlighting}[]
\CommentTok{# Analisando dados por agrupamentos}
\NormalTok{df }\OperatorTok\StringTok{ }\KeywordTok{group_by}\NormalTok{(ip) }\OperatorTok\StringTok{ }\KeywordTok{tally}\NormalTok{()}
\end{Highlighting}
\end{Shaded}

\begin{verbatim}
## # A tibble: 34,857 x 2
##       ip     n
##    <int> <int>
##  1     9     1
##  2    10     3
##  3    19     1
##  4    20     4
##  5    25     1
##  6    27     5
##  7    31     1
##  8    33     1
##  9    36     3
## 10    59     3
## # ... with 34,847 more rows
\end{verbatim}

\begin{Shaded}
\begin{Highlighting}[]
\NormalTok{df }\OperatorTok\StringTok{ }\KeywordTok{group_by}\NormalTok{(ip, click_day) }\OperatorTok\StringTok{ }\KeywordTok{tally}\NormalTok{()}
\end{Highlighting}
\end{Shaded}

\begin{verbatim}
## # A tibble: 55,454 x 3
## # Groups:   ip [34,857]
##       ip click_day     n
##    <int>     <int> <int>
##  1     9         7     1
##  2    10         7     2
##  3    10         8     1
##  4    19         8     1
##  5    20         8     3
##  6    20         9     1
##  7    25         7     1
##  8    27         7     1
##  9    27         8     2
## 10    27         9     2
## # ... with 55,444 more rows
\end{verbatim}

\begin{Shaded}
\begin{Highlighting}[]
\CommentTok{# Adicionando nova coluna no dataframe}
\NormalTok{df_count_ip <-}\StringTok{ }\NormalTok{df }\OperatorTok\StringTok{ }
\StringTok{  }\KeywordTok{count}\NormalTok{(ip, }\DataTypeTok{sort =} \OtherTok{TRUE}\NormalTok{, }\DataTypeTok{name =} \StringTok{"ip_count"}\NormalTok{)}

\NormalTok{df <-}\StringTok{ }\KeywordTok{merge}\NormalTok{(df, df_count_ip, }\DataTypeTok{by=}\KeywordTok{c}\NormalTok{(}\StringTok{"ip"}\NormalTok{))}
\KeywordTok{rm}\NormalTok{(}\StringTok{'df_count_ip'}\NormalTok{)}

\NormalTok{df_count_ip_day <-}\StringTok{ }\NormalTok{df }\OperatorTok\StringTok{ }
\StringTok{  }\KeywordTok{count}\NormalTok{(ip, click_day, }\DataTypeTok{sort =} \OtherTok{TRUE}\NormalTok{, }\DataTypeTok{name =} \StringTok{"ip_day_count"}\NormalTok{)}

\NormalTok{df <-}\StringTok{ }\KeywordTok{merge}\NormalTok{(df, df_count_ip_day, }\DataTypeTok{by=}\KeywordTok{c}\NormalTok{(}\StringTok{'ip'}\NormalTok{,}\StringTok{'click_day'}\NormalTok{))}
\KeywordTok{rm}\NormalTok{(}\StringTok{'df_count_ip_day'}\NormalTok{)}

\KeywordTok{str}\NormalTok{(df)}
\end{Highlighting}
\end{Shaded}

\begin{verbatim}
## 'data.frame':    100000 obs. of  10 variables:
##  $ ip           : int  10 10 10 1000 100002 100005 100005 100009 100013 100013 ...
##  $ click_day    : int  7 7 8 7 8 7 9 8 8 9 ...
##  $ app          : int  11 12 18 12 3 2 9 64 3 13 ...
##  $ device       : int  1 1 1 1 1 1 1 1 1 1 ...
##  $ os           : int  22 19 13 19 41 17 19 18 41 10 ...
##  $ channel      : int  319 140 107 178 280 219 232 459 442 477 ...
##  $ is_attributed: Factor w/ 2 levels "0","1": 1 1 1 1 1 1 1 1 1 1 ...
##  $ click_hour   : int  1 7 11 13 2 3 14 9 5 7 ...
##  $ ip_count     : int  3 3 3 1 1 2 2 1 2 2 ...
##  $ ip_day_count : int  2 2 1 1 1 1 1 1 1 1 ...
\end{verbatim}

\begin{Shaded}
\begin{Highlighting}[]
\CommentTok{# View(df)}

\CommentTok{# Normalizar as variaveis numericas }
\NormalTok{cols <-}\StringTok{ }\KeywordTok{c}\NormalTok{(}\StringTok{'ip'}\NormalTok{, }\StringTok{'app'}\NormalTok{, }\StringTok{'device'}\NormalTok{, }\StringTok{'os'}\NormalTok{, }\StringTok{'channel'}\NormalTok{,}\StringTok{'click_day'}\NormalTok{,}\StringTok{'click_hour'}\NormalTok{,}\StringTok{'ip_count'}\NormalTok{, }\StringTok{'ip_day_count'}\NormalTok{) }
\NormalTok{df[, cols] <-}\StringTok{ }\KeywordTok{scale}\NormalTok{(df[, cols])}

\CommentTok{# Verificando overfitting dos dados}
\CommentTok{# 99.773 registros indicam que o app nao foi baixado}
\CommentTok{# 227 registros indicam que o app foi baixado}
\KeywordTok{table}\NormalTok{(df}\OperatorTok{$}\NormalTok{is_attributed)}
\end{Highlighting}
\end{Shaded}

\begin{verbatim}
## 
##     0     1 
## 99773   227
\end{verbatim}

\begin{Shaded}
\begin{Highlighting}[]
\KeywordTok{prop.table}\NormalTok{(}\KeywordTok{table}\NormalTok{(df}\OperatorTok{$}\NormalTok{is_attributed))}
\end{Highlighting}
\end{Shaded}

\begin{verbatim}
## 
##       0       1 
## 0.99773 0.00227
\end{verbatim}

\begin{Shaded}
\begin{Highlighting}[]
\CommentTok{# Feature Selection (Selecao de Variaveis)}

\CommentTok{# Carregando os Pacotes}
\KeywordTok{library}\NormalTok{(ROSE)}
\end{Highlighting}
\end{Shaded}

\begin{verbatim}
## Loaded ROSE 0.0-3
\end{verbatim}

\begin{Shaded}
\begin{Highlighting}[]
\KeywordTok{library}\NormalTok{(caret)}
\end{Highlighting}
\end{Shaded}

\begin{verbatim}
## Loading required package: lattice
\end{verbatim}

\begin{verbatim}
## Loading required package: ggplot2
\end{verbatim}

\begin{verbatim}
## Registered S3 methods overwritten by 'ggplot2':
##   method         from 
##   [.quosures     rlang
##   c.quosures     rlang
##   print.quosures rlang
\end{verbatim}

\begin{Shaded}
\begin{Highlighting}[]
\KeywordTok{library}\NormalTok{(e1071)}
\KeywordTok{library}\NormalTok{(rpart)}

\CommentTok{# Gerando dados de treino e de teste}
\NormalTok{splits <-}\StringTok{ }\KeywordTok{createDataPartition}\NormalTok{(df}\OperatorTok{$}\NormalTok{is_attributed, }\DataTypeTok{p=}\FloatTok{0.7}\NormalTok{, }\DataTypeTok{list=}\OtherTok{FALSE}\NormalTok{)}

\CommentTok{# Separando os dados}
\NormalTok{dados_treino <-}\StringTok{ }\NormalTok{df[ splits,]}
\NormalTok{dados_teste <-}\StringTok{ }\NormalTok{df[}\OperatorTok{-}\NormalTok{splits,]}

\CommentTok{# Verificando o numero de linhas}
\KeywordTok{nrow}\NormalTok{(dados_treino)}
\end{Highlighting}
\end{Shaded}

\begin{verbatim}
## [1] 70001
\end{verbatim}

\begin{Shaded}
\begin{Highlighting}[]
\KeywordTok{nrow}\NormalTok{(dados_teste)}
\end{Highlighting}
\end{Shaded}

\begin{verbatim}
## [1] 29999
\end{verbatim}

\begin{Shaded}
\begin{Highlighting}[]
\CommentTok{# Treinando o modelo usando Baive Bayes e fazendo predicoes}
\CommentTok{## devido ao problema de overfitting, o resultado esta tendencioso}
\CommentTok{## necessario corrigir o problema de overfitting}
\NormalTok{modeloNB <-}\StringTok{ }\KeywordTok{naiveBayes}\NormalTok{(is_attributed }\OperatorTok{~}\NormalTok{. , }\DataTypeTok{data=}\NormalTok{dados_treino)}
\NormalTok{predNB <-}\StringTok{ }\KeywordTok{predict}\NormalTok{(modeloNB, dados_teste)}
\KeywordTok{confusionMatrix}\NormalTok{(predNB, dados_teste}\OperatorTok{$}\NormalTok{is_attributed)}
\end{Highlighting}
\end{Shaded}

\begin{verbatim}
## Confusion Matrix and Statistics
## 
##           Reference
## Prediction     0     1
##          0 29146    60
##          1   785     8
##                                           
##                Accuracy : 0.9718          
##                  95% CI : (0.9699, 0.9737)
##     No Information Rate : 0.9977          
##     P-Value [Acc > NIR] : 1               
##                                           
##                   Kappa : 0.0145          
##                                           
##  Mcnemar's Test P-Value : <2e-16          
##                                           
##             Sensitivity : 0.97377         
##             Specificity : 0.11765         
##          Pos Pred Value : 0.99795         
##          Neg Pred Value : 0.01009         
##              Prevalence : 0.99773         
##          Detection Rate : 0.97157         
##    Detection Prevalence : 0.97357         
##       Balanced Accuracy : 0.54571         
##                                           
##        'Positive' Class : 0               
## 
\end{verbatim}

\begin{Shaded}
\begin{Highlighting}[]
\CommentTok{# AUC}
\KeywordTok{roc.curve}\NormalTok{(dados_teste}\OperatorTok{$}\NormalTok{is_attributed, predNB)}
\end{Highlighting}
\end{Shaded}

\includegraphics{Projeto01_files/figure-latex/unnamed-chunk-1-1.pdf}

\begin{verbatim}
## Area under the curve (AUC): 0.546
\end{verbatim}

\begin{Shaded}
\begin{Highlighting}[]
\CommentTok{# Resolvendo problema de Overfitting usando pacote ROSE}
\CommentTok{#over sampling}
\NormalTok{dados_treino_new <-}\StringTok{ }\KeywordTok{ROSE}\NormalTok{(is_attributed }\OperatorTok{~}\StringTok{ }\NormalTok{. , }\DataTypeTok{data=}\NormalTok{dados_treino)}\OperatorTok{$}\NormalTok{data}
\KeywordTok{table}\NormalTok{(dados_treino_new}\OperatorTok{$}\NormalTok{is_attributed)}
\end{Highlighting}
\end{Shaded}

\begin{verbatim}
## 
##     0     1 
## 34958 35043
\end{verbatim}

\begin{Shaded}
\begin{Highlighting}[]
\KeywordTok{prop.table}\NormalTok{(}\KeywordTok{table}\NormalTok{(dados_treino_new}\OperatorTok{$}\NormalTok{is_attributed))}
\end{Highlighting}
\end{Shaded}

\begin{verbatim}
## 
##         0         1 
## 0.4993929 0.5006071
\end{verbatim}

\begin{Shaded}
\begin{Highlighting}[]
\CommentTok{# Treinando um novo modelo com os novos dados de treino }
\NormalTok{modeloNB_v2 <-}\StringTok{ }\KeywordTok{naiveBayes}\NormalTok{(is_attributed }\OperatorTok{~}\StringTok{ }\NormalTok{. , }\DataTypeTok{data=}\NormalTok{dados_treino_new)}
\NormalTok{predNB_v2 <-}\StringTok{ }\KeywordTok{predict}\NormalTok{(modeloNB_v2, dados_teste)}
\KeywordTok{confusionMatrix}\NormalTok{(predNB_v2, dados_teste}\OperatorTok{$}\NormalTok{is_attributed)}
\end{Highlighting}
\end{Shaded}

\begin{verbatim}
## Confusion Matrix and Statistics
## 
##           Reference
## Prediction     0     1
##          0 26782    32
##          1  3149    36
##                                           
##                Accuracy : 0.894           
##                  95% CI : (0.8904, 0.8974)
##     No Information Rate : 0.9977          
##     P-Value [Acc > NIR] : 1               
##                                           
##                   Kappa : 0.0178          
##                                           
##  Mcnemar's Test P-Value : <2e-16          
##                                           
##             Sensitivity : 0.8948          
##             Specificity : 0.5294          
##          Pos Pred Value : 0.9988          
##          Neg Pred Value : 0.0113          
##              Prevalence : 0.9977          
##          Detection Rate : 0.8928          
##    Detection Prevalence : 0.8938          
##       Balanced Accuracy : 0.7121          
##                                           
##        'Positive' Class : 0               
## 
\end{verbatim}

\begin{Shaded}
\begin{Highlighting}[]
\CommentTok{# AUC}
\KeywordTok{roc.curve}\NormalTok{(dados_teste}\OperatorTok{$}\NormalTok{is_attributed, predNB_v2)}
\end{Highlighting}
\end{Shaded}

\includegraphics{Projeto01_files/figure-latex/unnamed-chunk-1-2.pdf}

\begin{verbatim}
## Area under the curve (AUC): 0.712
\end{verbatim}

\begin{Shaded}
\begin{Highlighting}[]
\CommentTok{#AUC ROSE}
\NormalTok{ROSE.holdout <-}\StringTok{ }\KeywordTok{ROSE.eval}\NormalTok{(is_attributed }\OperatorTok{~}\StringTok{ }\NormalTok{., }
                          \DataTypeTok{data =}\NormalTok{ dados_treino_new, }
                          \DataTypeTok{learner =}\NormalTok{ rpart, }
                          \DataTypeTok{method.assess =} \StringTok{"holdout"}\NormalTok{, }
                          \DataTypeTok{extr.pred =} \ControlFlowTok{function}\NormalTok{(obj)obj[,}\DecValTok{2}\NormalTok{])}
\NormalTok{ROSE.holdout}
\end{Highlighting}
\end{Shaded}

\begin{verbatim}
## 
## Call: 
## ROSE.eval(formula = is_attributed ~ ., data = dados_treino_new, 
##     learner = rpart, extr.pred = function(obj) obj[, 2], method.assess = "holdout")
## 
## Holdout estimate of auc: 0.839
\end{verbatim}

\begin{Shaded}
\begin{Highlighting}[]
\CommentTok{# Análise de Correlação }

\CommentTok{# Carregando os Pacotes}
\KeywordTok{library}\NormalTok{(corrplot)}
\end{Highlighting}
\end{Shaded}

\begin{verbatim}
## corrplot 0.84 loaded
\end{verbatim}

\begin{Shaded}
\begin{Highlighting}[]
\KeywordTok{library}\NormalTok{(corrgram)}
\end{Highlighting}
\end{Shaded}

\begin{verbatim}
## Registered S3 method overwritten by 'seriation':
##   method         from 
##   reorder.hclust gclus
\end{verbatim}

\begin{verbatim}
## 
## Attaching package: 'corrgram'
\end{verbatim}

\begin{verbatim}
## The following object is masked from 'package:lattice':
## 
##     panel.fill
\end{verbatim}

\begin{Shaded}
\begin{Highlighting}[]
\CommentTok{# obtendo somente as colunas numericas}
\NormalTok{colunas_numericas <-}\StringTok{ }\KeywordTok{sapply}\NormalTok{(dados_treino_new, is.numeric)}
\NormalTok{colunas_numericas}
\end{Highlighting}
\end{Shaded}

\begin{verbatim}
##            ip     click_day           app        device            os 
##          TRUE          TRUE          TRUE          TRUE          TRUE 
##       channel is_attributed    click_hour      ip_count  ip_day_count 
##          TRUE         FALSE          TRUE          TRUE          TRUE
\end{verbatim}

\begin{Shaded}
\begin{Highlighting}[]
\CommentTok{# Filtrando as colunas numericas para correlacao}
\NormalTok{data_cor <-}\StringTok{ }\KeywordTok{cor}\NormalTok{(dados_treino_new[,colunas_numericas])}
\NormalTok{data_cor}
\end{Highlighting}
\end{Shaded}

\begin{verbatim}
##                       ip     click_day          app       device
## ip            1.00000000  0.1579999514  0.117774227  0.014162298
## click_day     0.15799995  1.0000000000 -0.090524101 -0.007219469
## app           0.11777423 -0.0905241012  1.000000000  0.052012828
## device        0.01416230 -0.0072194694  0.052012828  1.000000000
## os            0.10564884  0.0002437316  0.164313547  0.368893662
## channel      -0.09770644 -0.0025776600 -0.114813572 -0.010212315
## click_hour    0.07439546 -0.1659673525  0.030791459  0.015326838
## ip_count     -0.17089509 -0.0533412492  0.000721418  0.006888774
## ip_day_count -0.15911326 -0.0253762664 -0.029899133  0.007306874
##                         os     channel  click_hour     ip_count
## ip            0.1056488437 -0.09770644  0.07439546 -0.170895091
## click_day     0.0002437316 -0.00257766 -0.16596735 -0.053341249
## app           0.1643135472 -0.11481357  0.03079146  0.000721418
## device        0.3688936620 -0.01021232  0.01532684  0.006888774
## os            1.0000000000 -0.02152527  0.04368311 -0.002686173
## channel      -0.0215252650  1.00000000  0.10139260  0.024710893
## click_hour    0.0436831139  0.10139260  1.00000000  0.120536429
## ip_count     -0.0026861735  0.02471089  0.12053643  1.000000000
## ip_day_count -0.0041132779  0.03340135  0.08899555  0.710153056
##              ip_day_count
## ip           -0.159113261
## click_day    -0.025376266
## app          -0.029899133
## device        0.007306874
## os           -0.004113278
## channel       0.033401351
## click_hour    0.088995546
## ip_count      0.710153056
## ip_day_count  1.000000000
\end{verbatim}

\begin{Shaded}
\begin{Highlighting}[]
\KeywordTok{head}\NormalTok{(data_cor)}
\end{Highlighting}
\end{Shaded}

\begin{verbatim}
##                    ip     click_day         app       device            os
## ip         1.00000000  0.1579999514  0.11777423  0.014162298  0.1056488437
## click_day  0.15799995  1.0000000000 -0.09052410 -0.007219469  0.0002437316
## app        0.11777423 -0.0905241012  1.00000000  0.052012828  0.1643135472
## device     0.01416230 -0.0072194694  0.05201283  1.000000000  0.3688936620
## os         0.10564884  0.0002437316  0.16431355  0.368893662  1.0000000000
## channel   -0.09770644 -0.0025776600 -0.11481357 -0.010212315 -0.0215252650
##               channel  click_hour     ip_count ip_day_count
## ip        -0.09770644  0.07439546 -0.170895091 -0.159113261
## click_day -0.00257766 -0.16596735 -0.053341249 -0.025376266
## app       -0.11481357  0.03079146  0.000721418 -0.029899133
## device    -0.01021232  0.01532684  0.006888774  0.007306874
## os        -0.02152527  0.04368311 -0.002686173 -0.004113278
## channel    1.00000000  0.10139260  0.024710893  0.033401351
\end{verbatim}

\begin{Shaded}
\begin{Highlighting}[]
\CommentTok{# Criando um corrplot}
\KeywordTok{corrplot}\NormalTok{(data_cor, }\DataTypeTok{method =} \StringTok{'color'}\NormalTok{)}
\end{Highlighting}
\end{Shaded}

\includegraphics{Projeto01_files/figure-latex/unnamed-chunk-1-3.pdf}

\begin{Shaded}
\begin{Highlighting}[]
\CommentTok{# Criando um corrgram}
\KeywordTok{corrgram}\NormalTok{(dados_treino_new, }\DataTypeTok{order=}\OtherTok{TRUE}\NormalTok{, }\DataTypeTok{lower.panel =}\NormalTok{ panel.shade,}
         \DataTypeTok{upper.panel =}\NormalTok{ panel.pie, }\DataTypeTok{text.panel =}\NormalTok{ panel.txt)}
\end{Highlighting}
\end{Shaded}

\includegraphics{Projeto01_files/figure-latex/unnamed-chunk-1-4.pdf}

\begin{Shaded}
\begin{Highlighting}[]
\CommentTok{# Cria um modelo preditivo usando randomForest}

\CommentTok{# Carregando os Pacotes}
\KeywordTok{library}\NormalTok{(randomForest)}
\end{Highlighting}
\end{Shaded}

\begin{verbatim}
## randomForest 4.6-14
\end{verbatim}

\begin{verbatim}
## Type rfNews() to see new features/changes/bug fixes.
\end{verbatim}

\begin{verbatim}
## 
## Attaching package: 'randomForest'
\end{verbatim}

\begin{verbatim}
## The following object is masked from 'package:ggplot2':
## 
##     margin
\end{verbatim}

\begin{verbatim}
## The following object is masked from 'package:dplyr':
## 
##     combine
\end{verbatim}

\begin{Shaded}
\begin{Highlighting}[]
\CommentTok{# Cria o modelo preditivo usando randomForest}
\NormalTok{modeloRF <-}\StringTok{ }\KeywordTok{randomForest}\NormalTok{(is_attributed }\OperatorTok{~}\StringTok{ }\NormalTok{., }
                         \DataTypeTok{data =}\NormalTok{ dados_treino_new,}
                         \DataTypeTok{ntree =} \DecValTok{40}\NormalTok{, }
                         \DataTypeTok{nodesize =} \DecValTok{5}\NormalTok{)}
\KeywordTok{print}\NormalTok{(modeloRF)}
\end{Highlighting}
\end{Shaded}

\begin{verbatim}
## 
## Call:
##  randomForest(formula = is_attributed ~ ., data = dados_treino_new,      ntree = 40, nodesize = 5) 
##                Type of random forest: classification
##                      Number of trees: 40
## No. of variables tried at each split: 3
## 
##         OOB estimate of  error rate: 10.3%
## Confusion matrix:
##       0     1 class.error
## 0 31761  3197   0.0914526
## 1  4011 31032   0.1144594
\end{verbatim}

\begin{Shaded}
\begin{Highlighting}[]
\CommentTok{# Previsões com um modelo de classificação baseado em randomForest}

\CommentTok{# Gerando previsões nos dados de teste}
\NormalTok{previsoes <-}\StringTok{ }\KeywordTok{data.frame}\NormalTok{(}\DataTypeTok{observado =}\NormalTok{ dados_teste}\OperatorTok{$}\NormalTok{is_attributed,}
                        \DataTypeTok{previsto =} \KeywordTok{predict}\NormalTok{(modeloRF, }\DataTypeTok{newdata =}\NormalTok{ dados_teste))}


\CommentTok{# Visualizando o resultado}
\CommentTok{# View(previsoes)}

\CommentTok{# Calculando a Confusion Matrix em R}

\CommentTok{# Carregando os Pacotes}
\KeywordTok{library}\NormalTok{(ROCR)}
\end{Highlighting}
\end{Shaded}

\begin{verbatim}
## Loading required package: gplots
\end{verbatim}

\begin{verbatim}
## 
## Attaching package: 'gplots'
\end{verbatim}

\begin{verbatim}
## The following object is masked from 'package:stats':
## 
##     lowess
\end{verbatim}

\begin{Shaded}
\begin{Highlighting}[]
\CommentTok{# Gerando as classes de dados}
\NormalTok{class1 <-}\StringTok{ }\KeywordTok{predict}\NormalTok{(modeloRF, }\DataTypeTok{newdata =}\NormalTok{ dados_teste, }\DataTypeTok{type =} \StringTok{'prob'}\NormalTok{)}
\NormalTok{class2 <-}\StringTok{ }\NormalTok{dados_teste}\OperatorTok{$}\NormalTok{is_attributed}

\CommentTok{# Gerando a curva ROC}
\NormalTok{pred <-}\StringTok{ }\KeywordTok{prediction}\NormalTok{(class1[,}\DecValTok{2}\NormalTok{], class2)}
\NormalTok{perf <-}\StringTok{ }\KeywordTok{performance}\NormalTok{(pred, }\StringTok{"tpr"}\NormalTok{,}\StringTok{"fpr"}\NormalTok{) }
\KeywordTok{plot}\NormalTok{(perf, }\DataTypeTok{col =} \KeywordTok{rainbow}\NormalTok{(}\DecValTok{10}\NormalTok{))}
\end{Highlighting}
\end{Shaded}

\includegraphics{Projeto01_files/figure-latex/unnamed-chunk-1-5.pdf}

\begin{Shaded}
\begin{Highlighting}[]
\CommentTok{# Gerando Confusion Matrix com o Caret}
\CommentTok{# Dataframes com valores observados e previstos}
\NormalTok{previsoes_v2 <-}\StringTok{ }\KeywordTok{data.frame}\NormalTok{(}\DataTypeTok{observado =}\NormalTok{ dados_teste}\OperatorTok{$}\NormalTok{is_attributed,}
                           \DataTypeTok{previsto =} \KeywordTok{predict}\NormalTok{(}\DataTypeTok{object =}\NormalTok{ modeloRF, }\DataTypeTok{newdata =}\NormalTok{ dados_teste))}

\KeywordTok{confusionMatrix}\NormalTok{(previsoes_v2}\OperatorTok{$}\NormalTok{observado, previsoes_v2}\OperatorTok{$}\NormalTok{previsto)}
\end{Highlighting}
\end{Shaded}

\begin{verbatim}
## Confusion Matrix and Statistics
## 
##           Reference
## Prediction     0     1
##          0 29190   741
##          1    40    28
##                                           
##                Accuracy : 0.974           
##                  95% CI : (0.9721, 0.9757)
##     No Information Rate : 0.9744          
##     P-Value [Acc > NIR] : 0.6777          
##                                           
##                   Kappa : 0.063           
##                                           
##  Mcnemar's Test P-Value : <2e-16          
##                                           
##             Sensitivity : 0.99863         
##             Specificity : 0.03641         
##          Pos Pred Value : 0.97524         
##          Neg Pred Value : 0.41176         
##              Prevalence : 0.97437         
##          Detection Rate : 0.97303         
##    Detection Prevalence : 0.99773         
##       Balanced Accuracy : 0.51752         
##                                           
##        'Positive' Class : 0               
## 
\end{verbatim}

\begin{Shaded}
\begin{Highlighting}[]
\CommentTok{# Otimizando o Modelo preditivo}

\CommentTok{# Carregando os Pacotes}
\KeywordTok{library}\NormalTok{(rpart.plot)}

\CommentTok{# Criando uma Cost Function}
\NormalTok{Cost_func <-}\StringTok{ }\KeywordTok{matrix}\NormalTok{(}\KeywordTok{c}\NormalTok{(}\DecValTok{0}\NormalTok{, }\FloatTok{1.5}\NormalTok{, }\DecValTok{1}\NormalTok{, }\DecValTok{0}\NormalTok{), }\DataTypeTok{nrow =} \DecValTok{2}\NormalTok{, }\DataTypeTok{dimnames =} \KeywordTok{list}\NormalTok{(}\KeywordTok{c}\NormalTok{(}\StringTok{"1"}\NormalTok{, }\StringTok{"2"}\NormalTok{), }\KeywordTok{c}\NormalTok{(}\StringTok{"1"}\NormalTok{, }\StringTok{"2"}\NormalTok{)))}

\CommentTok{# Criando o Modelo usando rpart}
\NormalTok{modeloTree <-}\StringTok{ }\KeywordTok{rpart}\NormalTok{(is_attributed }\OperatorTok{~}\StringTok{ }\NormalTok{.,}
                    \DataTypeTok{data =}\NormalTok{ dados_treino_new,}
                    \DataTypeTok{method =} \StringTok{'class'}\NormalTok{,}
                    \DataTypeTok{parms =} \KeywordTok{list}\NormalTok{(}\DataTypeTok{loss =}\NormalTok{ Cost_func))}

\CommentTok{# Plot do modelo}
\KeywordTok{rpart.plot}\NormalTok{(modeloTree, }\DataTypeTok{fallen.leaves =} \OtherTok{FALSE}\NormalTok{, }\DataTypeTok{type =} \DecValTok{1}\NormalTok{)}
\end{Highlighting}
\end{Shaded}

\includegraphics{Projeto01_files/figure-latex/unnamed-chunk-1-6.pdf}

\begin{Shaded}
\begin{Highlighting}[]
\CommentTok{# Analisando Confusion Matrix}
\NormalTok{pred.tree <-}\StringTok{ }\KeywordTok{predict}\NormalTok{(modeloTree, }\DataTypeTok{type =} \StringTok{"class"}\NormalTok{)}
\KeywordTok{confusionMatrix}\NormalTok{(pred.tree, dados_treino_new}\OperatorTok{$}\NormalTok{is_attributed)}
\end{Highlighting}
\end{Shaded}

\begin{verbatim}
## Confusion Matrix and Statistics
## 
##           Reference
## Prediction     0     1
##          0 27677  4212
##          1  7281 30831
##                                           
##                Accuracy : 0.8358          
##                  95% CI : (0.8331, 0.8386)
##     No Information Rate : 0.5006          
##     P-Value [Acc > NIR] : < 2.2e-16       
##                                           
##                   Kappa : 0.6716          
##                                           
##  Mcnemar's Test P-Value : < 2.2e-16       
##                                           
##             Sensitivity : 0.7917          
##             Specificity : 0.8798          
##          Pos Pred Value : 0.8679          
##          Neg Pred Value : 0.8090          
##              Prevalence : 0.4994          
##          Detection Rate : 0.3954          
##    Detection Prevalence : 0.4556          
##       Balanced Accuracy : 0.8358          
##                                           
##        'Positive' Class : 0               
## 
\end{verbatim}


\end{document}
